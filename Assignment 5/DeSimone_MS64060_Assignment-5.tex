% Options for packages loaded elsewhere
\PassOptionsToPackage{unicode}{hyperref}
\PassOptionsToPackage{hyphens}{url}
%
\documentclass[
]{article}
\usepackage{amsmath,amssymb}
\usepackage{lmodern}
\usepackage{ifxetex,ifluatex}
\ifnum 0\ifxetex 1\fi\ifluatex 1\fi=0 % if pdftex
  \usepackage[T1]{fontenc}
  \usepackage[utf8]{inputenc}
  \usepackage{textcomp} % provide euro and other symbols
\else % if luatex or xetex
  \usepackage{unicode-math}
  \defaultfontfeatures{Scale=MatchLowercase}
  \defaultfontfeatures[\rmfamily]{Ligatures=TeX,Scale=1}
\fi
% Use upquote if available, for straight quotes in verbatim environments
\IfFileExists{upquote.sty}{\usepackage{upquote}}{}
\IfFileExists{microtype.sty}{% use microtype if available
  \usepackage[]{microtype}
  \UseMicrotypeSet[protrusion]{basicmath} % disable protrusion for tt fonts
}{}
\makeatletter
\@ifundefined{KOMAClassName}{% if non-KOMA class
  \IfFileExists{parskip.sty}{%
    \usepackage{parskip}
  }{% else
    \setlength{\parindent}{0pt}
    \setlength{\parskip}{6pt plus 2pt minus 1pt}}
}{% if KOMA class
  \KOMAoptions{parskip=half}}
\makeatother
\usepackage{xcolor}
\IfFileExists{xurl.sty}{\usepackage{xurl}}{} % add URL line breaks if available
\IfFileExists{bookmark.sty}{\usepackage{bookmark}}{\usepackage{hyperref}}
\hypersetup{
  pdftitle={DeSimone\_MS64060\_Assignment 5},
  pdfauthor={Heather DeSimone},
  hidelinks,
  pdfcreator={LaTeX via pandoc}}
\urlstyle{same} % disable monospaced font for URLs
\usepackage[margin=1in]{geometry}
\usepackage{color}
\usepackage{fancyvrb}
\newcommand{\VerbBar}{|}
\newcommand{\VERB}{\Verb[commandchars=\\\{\}]}
\DefineVerbatimEnvironment{Highlighting}{Verbatim}{commandchars=\\\{\}}
% Add ',fontsize=\small' for more characters per line
\usepackage{framed}
\definecolor{shadecolor}{RGB}{248,248,248}
\newenvironment{Shaded}{\begin{snugshade}}{\end{snugshade}}
\newcommand{\AlertTok}[1]{\textcolor[rgb]{0.94,0.16,0.16}{#1}}
\newcommand{\AnnotationTok}[1]{\textcolor[rgb]{0.56,0.35,0.01}{\textbf{\textit{#1}}}}
\newcommand{\AttributeTok}[1]{\textcolor[rgb]{0.77,0.63,0.00}{#1}}
\newcommand{\BaseNTok}[1]{\textcolor[rgb]{0.00,0.00,0.81}{#1}}
\newcommand{\BuiltInTok}[1]{#1}
\newcommand{\CharTok}[1]{\textcolor[rgb]{0.31,0.60,0.02}{#1}}
\newcommand{\CommentTok}[1]{\textcolor[rgb]{0.56,0.35,0.01}{\textit{#1}}}
\newcommand{\CommentVarTok}[1]{\textcolor[rgb]{0.56,0.35,0.01}{\textbf{\textit{#1}}}}
\newcommand{\ConstantTok}[1]{\textcolor[rgb]{0.00,0.00,0.00}{#1}}
\newcommand{\ControlFlowTok}[1]{\textcolor[rgb]{0.13,0.29,0.53}{\textbf{#1}}}
\newcommand{\DataTypeTok}[1]{\textcolor[rgb]{0.13,0.29,0.53}{#1}}
\newcommand{\DecValTok}[1]{\textcolor[rgb]{0.00,0.00,0.81}{#1}}
\newcommand{\DocumentationTok}[1]{\textcolor[rgb]{0.56,0.35,0.01}{\textbf{\textit{#1}}}}
\newcommand{\ErrorTok}[1]{\textcolor[rgb]{0.64,0.00,0.00}{\textbf{#1}}}
\newcommand{\ExtensionTok}[1]{#1}
\newcommand{\FloatTok}[1]{\textcolor[rgb]{0.00,0.00,0.81}{#1}}
\newcommand{\FunctionTok}[1]{\textcolor[rgb]{0.00,0.00,0.00}{#1}}
\newcommand{\ImportTok}[1]{#1}
\newcommand{\InformationTok}[1]{\textcolor[rgb]{0.56,0.35,0.01}{\textbf{\textit{#1}}}}
\newcommand{\KeywordTok}[1]{\textcolor[rgb]{0.13,0.29,0.53}{\textbf{#1}}}
\newcommand{\NormalTok}[1]{#1}
\newcommand{\OperatorTok}[1]{\textcolor[rgb]{0.81,0.36,0.00}{\textbf{#1}}}
\newcommand{\OtherTok}[1]{\textcolor[rgb]{0.56,0.35,0.01}{#1}}
\newcommand{\PreprocessorTok}[1]{\textcolor[rgb]{0.56,0.35,0.01}{\textit{#1}}}
\newcommand{\RegionMarkerTok}[1]{#1}
\newcommand{\SpecialCharTok}[1]{\textcolor[rgb]{0.00,0.00,0.00}{#1}}
\newcommand{\SpecialStringTok}[1]{\textcolor[rgb]{0.31,0.60,0.02}{#1}}
\newcommand{\StringTok}[1]{\textcolor[rgb]{0.31,0.60,0.02}{#1}}
\newcommand{\VariableTok}[1]{\textcolor[rgb]{0.00,0.00,0.00}{#1}}
\newcommand{\VerbatimStringTok}[1]{\textcolor[rgb]{0.31,0.60,0.02}{#1}}
\newcommand{\WarningTok}[1]{\textcolor[rgb]{0.56,0.35,0.01}{\textbf{\textit{#1}}}}
\usepackage{graphicx}
\makeatletter
\def\maxwidth{\ifdim\Gin@nat@width>\linewidth\linewidth\else\Gin@nat@width\fi}
\def\maxheight{\ifdim\Gin@nat@height>\textheight\textheight\else\Gin@nat@height\fi}
\makeatother
% Scale images if necessary, so that they will not overflow the page
% margins by default, and it is still possible to overwrite the defaults
% using explicit options in \includegraphics[width, height, ...]{}
\setkeys{Gin}{width=\maxwidth,height=\maxheight,keepaspectratio}
% Set default figure placement to htbp
\makeatletter
\def\fps@figure{htbp}
\makeatother
\setlength{\emergencystretch}{3em} % prevent overfull lines
\providecommand{\tightlist}{%
  \setlength{\itemsep}{0pt}\setlength{\parskip}{0pt}}
\setcounter{secnumdepth}{-\maxdimen} % remove section numbering
\ifluatex
  \usepackage{selnolig}  % disable illegal ligatures
\fi

\title{DeSimone\_MS64060\_Assignment 5}
\author{Heather DeSimone}
\date{4/9/2022}

\begin{document}
\maketitle

\#\#First I have loaded in my data frame and removed the cereals that
are missing information.

\begin{Shaded}
\begin{Highlighting}[]
\FunctionTok{library}\NormalTok{(dplyr)}
\end{Highlighting}
\end{Shaded}

\begin{verbatim}
## 
## Attaching package: 'dplyr'
\end{verbatim}

\begin{verbatim}
## The following objects are masked from 'package:stats':
## 
##     filter, lag
\end{verbatim}

\begin{verbatim}
## The following objects are masked from 'package:base':
## 
##     intersect, setdiff, setequal, union
\end{verbatim}

\begin{Shaded}
\begin{Highlighting}[]
\FunctionTok{library}\NormalTok{(caret)}
\end{Highlighting}
\end{Shaded}

\begin{verbatim}
## Warning: package 'caret' was built under R version 4.1.2
\end{verbatim}

\begin{verbatim}
## Loading required package: ggplot2
\end{verbatim}

\begin{verbatim}
## Warning: package 'ggplot2' was built under R version 4.1.2
\end{verbatim}

\begin{verbatim}
## Loading required package: lattice
\end{verbatim}

\begin{Shaded}
\begin{Highlighting}[]
\FunctionTok{library}\NormalTok{(class)}
\FunctionTok{library}\NormalTok{(ISLR)}
\end{Highlighting}
\end{Shaded}

\begin{verbatim}
## Warning: package 'ISLR' was built under R version 4.1.1
\end{verbatim}

\begin{Shaded}
\begin{Highlighting}[]
\FunctionTok{library}\NormalTok{(tidyverse)}
\end{Highlighting}
\end{Shaded}

\begin{verbatim}
## Warning: package 'tidyverse' was built under R version 4.1.3
\end{verbatim}

\begin{verbatim}
## -- Attaching packages --------------------------------------- tidyverse 1.3.1 --
\end{verbatim}

\begin{verbatim}
## v tibble  3.1.2     v purrr   0.3.4
## v tidyr   1.1.4     v stringr 1.4.0
## v readr   2.1.2     v forcats 0.5.1
\end{verbatim}

\begin{verbatim}
## Warning: package 'tidyr' was built under R version 4.1.2
\end{verbatim}

\begin{verbatim}
## Warning: package 'readr' was built under R version 4.1.3
\end{verbatim}

\begin{verbatim}
## Warning: package 'stringr' was built under R version 4.1.2
\end{verbatim}

\begin{verbatim}
## Warning: package 'forcats' was built under R version 4.1.3
\end{verbatim}

\begin{verbatim}
## -- Conflicts ------------------------------------------ tidyverse_conflicts() --
## x dplyr::filter() masks stats::filter()
## x dplyr::lag()    masks stats::lag()
## x purrr::lift()   masks caret::lift()
\end{verbatim}

\begin{Shaded}
\begin{Highlighting}[]
\FunctionTok{library}\NormalTok{(factoextra)}
\end{Highlighting}
\end{Shaded}

\begin{verbatim}
## Warning: package 'factoextra' was built under R version 4.1.3
\end{verbatim}

\begin{verbatim}
## Welcome! Want to learn more? See two factoextra-related books at https://goo.gl/ve3WBa
\end{verbatim}

\begin{Shaded}
\begin{Highlighting}[]
\FunctionTok{library}\NormalTok{(stats)}
\NormalTok{DF}\OtherTok{=}\FunctionTok{read.csv}\NormalTok{(}\StringTok{"C:/Users/hdesi/Desktop/MBA/Machine Learning/Cereals.csv"}\NormalTok{)}
\NormalTok{DF }\OtherTok{\textless{}{-}} \FunctionTok{na.omit}\NormalTok{(DF) }\DocumentationTok{\#\#Remove cereals missing data}
\NormalTok{DF}\SpecialCharTok{$}\NormalTok{mfr}\OtherTok{\textless{}{-}}\ConstantTok{NULL} \DocumentationTok{\#\#Not needed}
\NormalTok{DF}\SpecialCharTok{$}\NormalTok{type}\OtherTok{\textless{}{-}}\ConstantTok{NULL} \DocumentationTok{\#\#Not needed}
\FunctionTok{rownames}\NormalTok{(DF) }\OtherTok{\textless{}{-}}\NormalTok{ DF}\SpecialCharTok{$}\NormalTok{name }\DocumentationTok{\#\#Change row name to cereal name rather than numeric value}
\NormalTok{DF}\SpecialCharTok{$}\NormalTok{name}\OtherTok{\textless{}{-}}\ConstantTok{NULL}
\FunctionTok{head}\NormalTok{(DF)}
\end{Highlighting}
\end{Shaded}

\begin{verbatim}
##                           calories protein fat sodium fiber carbo sugars potass
## 100%_Bran                       70       4   1    130  10.0   5.0      6    280
## 100%_Natural_Bran              120       3   5     15   2.0   8.0      8    135
## All-Bran                        70       4   1    260   9.0   7.0      5    320
## All-Bran_with_Extra_Fiber       50       4   0    140  14.0   8.0      0    330
## Apple_Cinnamon_Cheerios        110       2   2    180   1.5  10.5     10     70
## Apple_Jacks                    110       2   0    125   1.0  11.0     14     30
##                           vitamins shelf weight cups   rating
## 100%_Bran                       25     3      1 0.33 68.40297
## 100%_Natural_Bran                0     3      1 1.00 33.98368
## All-Bran                        25     3      1 0.33 59.42551
## All-Bran_with_Extra_Fiber       25     3      1 0.50 93.70491
## Apple_Cinnamon_Cheerios         25     1      1 0.75 29.50954
## Apple_Jacks                     25     2      1 1.00 33.17409
\end{verbatim}

\begin{Shaded}
\begin{Highlighting}[]
\FunctionTok{sapply}\NormalTok{(DF, class) }\DocumentationTok{\#\#Making sure variables are numerical}
\end{Highlighting}
\end{Shaded}

\begin{verbatim}
##  calories   protein       fat    sodium     fiber     carbo    sugars    potass 
## "integer" "integer" "integer" "integer" "numeric" "numeric" "integer" "integer" 
##  vitamins     shelf    weight      cups    rating 
## "integer" "integer" "numeric" "numeric" "numeric"
\end{verbatim}

\#\#Creating data frame for normalization

\begin{Shaded}
\begin{Highlighting}[]
\NormalTok{DF.norm }\OtherTok{\textless{}{-}} \FunctionTok{data.frame}\NormalTok{(DF)}

\FunctionTok{head}\NormalTok{(DF.norm)}
\end{Highlighting}
\end{Shaded}

\begin{verbatim}
##                           calories protein fat sodium fiber carbo sugars potass
## 100%_Bran                       70       4   1    130  10.0   5.0      6    280
## 100%_Natural_Bran              120       3   5     15   2.0   8.0      8    135
## All-Bran                        70       4   1    260   9.0   7.0      5    320
## All-Bran_with_Extra_Fiber       50       4   0    140  14.0   8.0      0    330
## Apple_Cinnamon_Cheerios        110       2   2    180   1.5  10.5     10     70
## Apple_Jacks                    110       2   0    125   1.0  11.0     14     30
##                           vitamins shelf weight cups   rating
## 100%_Bran                       25     3      1 0.33 68.40297
## 100%_Natural_Bran                0     3      1 1.00 33.98368
## All-Bran                        25     3      1 0.33 59.42551
## All-Bran_with_Extra_Fiber       25     3      1 0.50 93.70491
## Apple_Cinnamon_Cheerios         25     1      1 0.75 29.50954
## Apple_Jacks                     25     2      1 1.00 33.17409
\end{verbatim}

\#\#I will perform hierarchical clustering using Euclidean Distance

\begin{Shaded}
\begin{Highlighting}[]
\NormalTok{DF.norm }\OtherTok{\textless{}{-}} \FunctionTok{scale}\NormalTok{(DF) }\DocumentationTok{\#\#Data normalization}
\NormalTok{DF.norm.Euclidean }\OtherTok{\textless{}{-}} \FunctionTok{dist}\NormalTok{(DF.norm, }\AttributeTok{method =} \StringTok{"euclidean"}\NormalTok{)}
\NormalTok{hc1 }\OtherTok{\textless{}{-}} \FunctionTok{hclust}\NormalTok{(DF.norm.Euclidean, }\AttributeTok{method =} \StringTok{"complete"}\NormalTok{)}

\FunctionTok{plot}\NormalTok{(hc1, }\AttributeTok{cex =}\NormalTok{ .}\DecValTok{6}\NormalTok{, }\AttributeTok{hang =} \SpecialCharTok{{-}}\DecValTok{1}\NormalTok{) }\DocumentationTok{\#\#Plotting the cluster Dendrogram using all variables still in dataset}
\end{Highlighting}
\end{Shaded}

\includegraphics{DeSimone_MS64060_Assignment-5_files/figure-latex/unnamed-chunk-4-1.pdf}

\#\#I will now use Agnes to compare clustering methods to find the best
one

\begin{Shaded}
\begin{Highlighting}[]
\FunctionTok{library}\NormalTok{(cluster)}
\NormalTok{hc\_single }\OtherTok{\textless{}{-}} \FunctionTok{agnes}\NormalTok{(DF.norm, }\AttributeTok{method =} \StringTok{"single"}\NormalTok{)}
\NormalTok{hc\_complete }\OtherTok{\textless{}{-}} \FunctionTok{agnes}\NormalTok{(DF.norm, }\AttributeTok{method =} \StringTok{"complete"}\NormalTok{)}
\NormalTok{hc\_average }\OtherTok{\textless{}{-}} \FunctionTok{agnes}\NormalTok{(DF.norm, }\AttributeTok{method =} \StringTok{"average"}\NormalTok{)}
\NormalTok{hc\_ward }\OtherTok{\textless{}{-}} \FunctionTok{agnes}\NormalTok{(DF.norm, }\AttributeTok{method =} \StringTok{"ward"}\NormalTok{) }\DocumentationTok{\#\#Ward is the best method}

\FunctionTok{print}\NormalTok{(hc\_single}\SpecialCharTok{$}\NormalTok{ac)}
\end{Highlighting}
\end{Shaded}

\begin{verbatim}
## [1] 0.6067859
\end{verbatim}

\begin{Shaded}
\begin{Highlighting}[]
\FunctionTok{print}\NormalTok{(hc\_complete}\SpecialCharTok{$}\NormalTok{ac)}
\end{Highlighting}
\end{Shaded}

\begin{verbatim}
## [1] 0.8353712
\end{verbatim}

\begin{Shaded}
\begin{Highlighting}[]
\FunctionTok{print}\NormalTok{(hc\_average}\SpecialCharTok{$}\NormalTok{ac)}
\end{Highlighting}
\end{Shaded}

\begin{verbatim}
## [1] 0.7766075
\end{verbatim}

\begin{Shaded}
\begin{Highlighting}[]
\FunctionTok{print}\NormalTok{(hc\_ward}\SpecialCharTok{$}\NormalTok{ac) }\DocumentationTok{\#\#closest to 1}
\end{Highlighting}
\end{Shaded}

\begin{verbatim}
## [1] 0.9046042
\end{verbatim}

\#\#I will now create my Agnes Dendrogram

\begin{Shaded}
\begin{Highlighting}[]
\FunctionTok{pltree}\NormalTok{(hc\_ward, }\AttributeTok{cex =} \FloatTok{0.6}\NormalTok{, }\AttributeTok{hang =} \SpecialCharTok{{-}}\DecValTok{1}\NormalTok{, }\AttributeTok{main =} \StringTok{"Dendogram of Agnes"}\NormalTok{)}
\FunctionTok{rect.hclust}\NormalTok{(hc\_ward, }\AttributeTok{k =} \DecValTok{4}\NormalTok{, }\AttributeTok{border =} \DecValTok{1}\SpecialCharTok{:}\DecValTok{4}\NormalTok{) }\DocumentationTok{\#\#4 clusters}
\end{Highlighting}
\end{Shaded}

\includegraphics{DeSimone_MS64060_Assignment-5_files/figure-latex/unnamed-chunk-6-1.pdf}

\#\#Now I want to cluster my data by unhealthy variables. For our
purposes, we will assume that cereals high in calories, fat, sugar, and
sodium are unhealthy.

\begin{Shaded}
\begin{Highlighting}[]
\NormalTok{DF.Unhealthy }\OtherTok{\textless{}{-}}\NormalTok{ DF[}\FunctionTok{c}\NormalTok{(}\DecValTok{1}\NormalTok{,}\DecValTok{3}\NormalTok{,}\DecValTok{4}\NormalTok{,}\DecValTok{7}\NormalTok{)] }\DocumentationTok{\#\#Calories, fat, sodium, sugar}
\FunctionTok{head}\NormalTok{(DF.Unhealthy)}
\end{Highlighting}
\end{Shaded}

\begin{verbatim}
##                           calories fat sodium sugars
## 100%_Bran                       70   1    130      6
## 100%_Natural_Bran              120   5     15      8
## All-Bran                        70   1    260      5
## All-Bran_with_Extra_Fiber       50   0    140      0
## Apple_Cinnamon_Cheerios        110   2    180     10
## Apple_Jacks                    110   0    125     14
\end{verbatim}

\#\#Finding best Agnes method

\begin{Shaded}
\begin{Highlighting}[]
\NormalTok{unhealthy\_single }\OtherTok{\textless{}{-}} \FunctionTok{agnes}\NormalTok{(DF.Unhealthy, }\AttributeTok{method =} \StringTok{"single"}\NormalTok{)}
\NormalTok{unhealthy\_complete }\OtherTok{\textless{}{-}} \FunctionTok{agnes}\NormalTok{(DF.Unhealthy, }\AttributeTok{method =} \StringTok{"complete"}\NormalTok{)}
\NormalTok{unhealthy\_average }\OtherTok{\textless{}{-}} \FunctionTok{agnes}\NormalTok{(DF.Unhealthy, }\AttributeTok{method =} \StringTok{"average"}\NormalTok{)}
\NormalTok{unhealthy\_ward }\OtherTok{\textless{}{-}} \FunctionTok{agnes}\NormalTok{(DF.Unhealthy, }\AttributeTok{method =} \StringTok{"ward"}\NormalTok{) }\DocumentationTok{\#\#Best method}

\FunctionTok{print}\NormalTok{(unhealthy\_single}\SpecialCharTok{$}\NormalTok{ac)}
\end{Highlighting}
\end{Shaded}

\begin{verbatim}
## [1] 0.7794119
\end{verbatim}

\begin{Shaded}
\begin{Highlighting}[]
\FunctionTok{print}\NormalTok{(unhealthy\_complete}\SpecialCharTok{$}\NormalTok{ac)}
\end{Highlighting}
\end{Shaded}

\begin{verbatim}
## [1] 0.967792
\end{verbatim}

\begin{Shaded}
\begin{Highlighting}[]
\FunctionTok{print}\NormalTok{(unhealthy\_average}\SpecialCharTok{$}\NormalTok{ac)}
\end{Highlighting}
\end{Shaded}

\begin{verbatim}
## [1] 0.9423144
\end{verbatim}

\begin{Shaded}
\begin{Highlighting}[]
\FunctionTok{print}\NormalTok{(unhealthy\_ward}\SpecialCharTok{$}\NormalTok{ac)}
\end{Highlighting}
\end{Shaded}

\begin{verbatim}
## [1] 0.9868955
\end{verbatim}

\#\#Ward was the best method for clustering. Now we will create our
dendograph to look at our clusters for unhealthy variables.

\begin{Shaded}
\begin{Highlighting}[]
\FunctionTok{pltree}\NormalTok{(unhealthy\_ward, }\AttributeTok{cex =} \FloatTok{0.6}\NormalTok{, }\AttributeTok{hang =} \SpecialCharTok{{-}}\DecValTok{1}\NormalTok{, }\AttributeTok{main =} \StringTok{"Dendograph Using Unhealthy Variables:Fat, Calories, Sugar \& Sodium"}\NormalTok{)}
\FunctionTok{rect.hclust}\NormalTok{(unhealthy\_ward, }\AttributeTok{k =} \DecValTok{4}\NormalTok{, }\AttributeTok{border =} \DecValTok{1}\SpecialCharTok{:}\DecValTok{4}\NormalTok{)}
\end{Highlighting}
\end{Shaded}

\includegraphics{DeSimone_MS64060_Assignment-5_files/figure-latex/unnamed-chunk-9-1.pdf}

\#\#So far, it looks like the healthiest cluster is cluster 1(black) and
the least healthy is cluster 4 (blue)

\#\#Now we will cluster based on healthy variables. Those high in
protein, fiber, and vitamins are most healthy.

\begin{Shaded}
\begin{Highlighting}[]
\NormalTok{DF.Healthy }\OtherTok{\textless{}{-}}\NormalTok{ DF[}\FunctionTok{c}\NormalTok{(}\DecValTok{2}\NormalTok{,}\DecValTok{5}\NormalTok{,}\DecValTok{9}\NormalTok{)] }\DocumentationTok{\#\#Protein, fiber, vitamins}
\FunctionTok{head}\NormalTok{(DF.Healthy)}
\end{Highlighting}
\end{Shaded}

\begin{verbatim}
##                           protein fiber vitamins
## 100%_Bran                       4  10.0       25
## 100%_Natural_Bran               3   2.0        0
## All-Bran                        4   9.0       25
## All-Bran_with_Extra_Fiber       4  14.0       25
## Apple_Cinnamon_Cheerios         2   1.5       25
## Apple_Jacks                     2   1.0       25
\end{verbatim}

\#\#Finding best Agnes method

\begin{Shaded}
\begin{Highlighting}[]
\NormalTok{healthy\_single }\OtherTok{\textless{}{-}} \FunctionTok{agnes}\NormalTok{(DF.Healthy, }\AttributeTok{method =} \StringTok{"single"}\NormalTok{)}
\NormalTok{healthy\_complete }\OtherTok{\textless{}{-}} \FunctionTok{agnes}\NormalTok{(DF.Healthy, }\AttributeTok{method =} \StringTok{"complete"}\NormalTok{)}
\NormalTok{healthy\_average }\OtherTok{\textless{}{-}} \FunctionTok{agnes}\NormalTok{(DF.Healthy, }\AttributeTok{method =} \StringTok{"average"}\NormalTok{)}
\NormalTok{healthy\_ward }\OtherTok{\textless{}{-}} \FunctionTok{agnes}\NormalTok{(DF.Healthy, }\AttributeTok{method =} \StringTok{"ward"}\NormalTok{)}\DocumentationTok{\#\#Best method}

\FunctionTok{print}\NormalTok{(healthy\_single}\SpecialCharTok{$}\NormalTok{ac)}
\end{Highlighting}
\end{Shaded}

\begin{verbatim}
## [1] 0.9950214
\end{verbatim}

\begin{Shaded}
\begin{Highlighting}[]
\FunctionTok{print}\NormalTok{(healthy\_complete}\SpecialCharTok{$}\NormalTok{ac)}
\end{Highlighting}
\end{Shaded}

\begin{verbatim}
## [1] 0.9957495
\end{verbatim}

\begin{Shaded}
\begin{Highlighting}[]
\FunctionTok{print}\NormalTok{(healthy\_average}\SpecialCharTok{$}\NormalTok{ac)}
\end{Highlighting}
\end{Shaded}

\begin{verbatim}
## [1] 0.9948298
\end{verbatim}

\begin{Shaded}
\begin{Highlighting}[]
\FunctionTok{print}\NormalTok{(healthy\_ward}\SpecialCharTok{$}\NormalTok{ac)}
\end{Highlighting}
\end{Shaded}

\begin{verbatim}
## [1] 0.9983455
\end{verbatim}

\#\#Ward was the best method for clustering. Now we will create our
dendograph to look at our clusters for healthy variables.

\begin{Shaded}
\begin{Highlighting}[]
\FunctionTok{pltree}\NormalTok{(healthy\_ward, }\AttributeTok{cex =} \FloatTok{0.6}\NormalTok{, }\AttributeTok{hang =} \SpecialCharTok{{-}}\DecValTok{1}\NormalTok{, }\AttributeTok{main =} \StringTok{"Dendograph Using Healthy Variables:Protein, Fiber \& Vitamins"}\NormalTok{)}
\FunctionTok{rect.hclust}\NormalTok{(healthy\_ward, }\AttributeTok{k =} \DecValTok{4}\NormalTok{, }\AttributeTok{border =} \DecValTok{1}\SpecialCharTok{:}\DecValTok{4}\NormalTok{)}
\end{Highlighting}
\end{Shaded}

\includegraphics{DeSimone_MS64060_Assignment-5_files/figure-latex/unnamed-chunk-12-1.pdf}

\#\#Clustering is a bit uneven, but it looks like cluster 1 (Black) is
the healthiest and there are repeat cereals in this healthy cluster that
were also in the healthy cluster is our last dendograph

\#\#Now we will look at all of our health related variables together.
Our cluster will consist of protein, fiber, vitamins, calories, fat,
sugar, and sodium

\begin{Shaded}
\begin{Highlighting}[]
\NormalTok{DF.TotalHealth }\OtherTok{\textless{}{-}}\NormalTok{ DF[}\FunctionTok{c}\NormalTok{(}\DecValTok{1}\NormalTok{,}\DecValTok{2}\NormalTok{,}\DecValTok{3}\NormalTok{,}\DecValTok{4}\NormalTok{,}\DecValTok{5}\NormalTok{,}\DecValTok{7}\NormalTok{,}\DecValTok{9}\NormalTok{)]}
\FunctionTok{head}\NormalTok{(DF.TotalHealth)}
\end{Highlighting}
\end{Shaded}

\begin{verbatim}
##                           calories protein fat sodium fiber sugars vitamins
## 100%_Bran                       70       4   1    130  10.0      6       25
## 100%_Natural_Bran              120       3   5     15   2.0      8        0
## All-Bran                        70       4   1    260   9.0      5       25
## All-Bran_with_Extra_Fiber       50       4   0    140  14.0      0       25
## Apple_Cinnamon_Cheerios        110       2   2    180   1.5     10       25
## Apple_Jacks                    110       2   0    125   1.0     14       25
\end{verbatim}

\#\#Finding best Agnes

\begin{Shaded}
\begin{Highlighting}[]
\NormalTok{TotalHealth\_single }\OtherTok{\textless{}{-}} \FunctionTok{agnes}\NormalTok{(DF.TotalHealth, }\AttributeTok{method =} \StringTok{"single"}\NormalTok{)}
\NormalTok{TotalHealth\_complete }\OtherTok{\textless{}{-}} \FunctionTok{agnes}\NormalTok{(DF.TotalHealth, }\AttributeTok{method =} \StringTok{"complete"}\NormalTok{)}
\NormalTok{TotalHealth\_average }\OtherTok{\textless{}{-}} \FunctionTok{agnes}\NormalTok{(DF.TotalHealth, }\AttributeTok{method =} \StringTok{"average"}\NormalTok{)}
\NormalTok{TotalHealth\_ward }\OtherTok{\textless{}{-}} \FunctionTok{agnes}\NormalTok{(DF.TotalHealth, }\AttributeTok{method =} \StringTok{"ward"}\NormalTok{) }\DocumentationTok{\#\#Best Method}

\FunctionTok{print}\NormalTok{(TotalHealth\_single}\SpecialCharTok{$}\NormalTok{ac)}
\end{Highlighting}
\end{Shaded}

\begin{verbatim}
## [1] 0.8615331
\end{verbatim}

\begin{Shaded}
\begin{Highlighting}[]
\FunctionTok{print}\NormalTok{(TotalHealth\_complete}\SpecialCharTok{$}\NormalTok{ac)}
\end{Highlighting}
\end{Shaded}

\begin{verbatim}
## [1] 0.9604174
\end{verbatim}

\begin{Shaded}
\begin{Highlighting}[]
\FunctionTok{print}\NormalTok{(TotalHealth\_average}\SpecialCharTok{$}\NormalTok{ac)}
\end{Highlighting}
\end{Shaded}

\begin{verbatim}
## [1] 0.9306789
\end{verbatim}

\begin{Shaded}
\begin{Highlighting}[]
\FunctionTok{print}\NormalTok{(TotalHealth\_ward}\SpecialCharTok{$}\NormalTok{ac)}
\end{Highlighting}
\end{Shaded}

\begin{verbatim}
## [1] 0.9837845
\end{verbatim}

\#\#Ward is again the best method. Now we will look at our dendograph in
which we clustered based on all health variables (good and bad)

\begin{Shaded}
\begin{Highlighting}[]
\FunctionTok{pltree}\NormalTok{(TotalHealth\_ward, }\AttributeTok{cex =} \FloatTok{0.6}\NormalTok{, }\AttributeTok{hang =} \SpecialCharTok{{-}}\DecValTok{1}\NormalTok{, }\AttributeTok{main =} \StringTok{"Dendograph:Fat, Cals, Sugars, Sodium, Protein, Fiber \& Vitamins"}\NormalTok{)}
\FunctionTok{rect.hclust}\NormalTok{(TotalHealth\_ward, }\AttributeTok{k =} \DecValTok{4}\NormalTok{, }\AttributeTok{border =} \DecValTok{1}\SpecialCharTok{:}\DecValTok{4}\NormalTok{)}
\end{Highlighting}
\end{Shaded}

\includegraphics{DeSimone_MS64060_Assignment-5_files/figure-latex/unnamed-chunk-15-1.pdf}

\#\#This dendograph looks very similar to the 1st one we did (With
unhealthy variables) \#\#Cluster 1 (black) looks to be the overall
healthiest cereals

\#\#I want to create more than 4 cluster using this same overall health
model

\begin{Shaded}
\begin{Highlighting}[]
\FunctionTok{pltree}\NormalTok{(TotalHealth\_ward, }\AttributeTok{cex =} \FloatTok{0.6}\NormalTok{, }\AttributeTok{hang =} \SpecialCharTok{{-}}\DecValTok{1}\NormalTok{, }\AttributeTok{main =} \StringTok{"Dendograph:Fat, Cals, Sugars, Sodium, Protein, Fiber \& Vitamins"}\NormalTok{)}
\FunctionTok{rect.hclust}\NormalTok{(TotalHealth\_ward, }\AttributeTok{k =} \DecValTok{8}\NormalTok{, }\AttributeTok{border =} \DecValTok{1}\SpecialCharTok{:}\DecValTok{4}\NormalTok{) }\DocumentationTok{\#\#8 clusters}
\end{Highlighting}
\end{Shaded}

\includegraphics{DeSimone_MS64060_Assignment-5_files/figure-latex/unnamed-chunk-16-1.pdf}

\#\#This 8 cluster model gives us a much more condensed list. With 100\%
bran in all of our dendograph healthy clusters, we will assume that the
cluster this cereal falls into is the healthiest group - based on sugar,
fat, calories, sodium, protein, fiber, and vitamins. The school could
use any of these dendographs to base their decision on, depending on
what they are looking for and what they consider healthy. Some people
feel that a diet low in calories, fat, sugar, and sodium is healthy even
if those foods are low in nutrients. Some people feel that a diet high
in vitamins, fiber, and protein are healthy even if they have higher
calories, fat, sugar. and sodium. I believe that the last dendograph
should be used (8 clusters) as it takes all of these variables into
consideration.

\end{document}
